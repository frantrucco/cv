\documentclass[10pt,a4paper,roman]{moderncv}

\moderncvstyle{classic}
% Options: casual, classic, oldstyle and banking
\moderncvcolor{purple}
% Options: blue, orange, green, red, purple, grey and black

\usepackage[pdftex,top=1.0cm,left=2.5cm,right=2.5cm,bottom=1.5cm]{geometry}
\usepackage[utf8]{inputenc}
\usepackage{setspace}

\firstname{Francisco Carlos}
\familyname{Trucco Dalmas}

\title{Lebenslauf}
\address{Herner Str. 155}{44809 Bochum, Germany}
\mobile{+43 0664 4666 289}

\email{francisco.trucco@mpi-sp.org}

\begin{document}

\makecvtitle

\begin{spacing}{1.05}
\section{Bildung}

\cventry{2014 -- Dez 2019}{Licenciatura en Ciencias de la
  Computación (Entspricht einem Diplomstudium der Informatik)}{Universidad Nacional de Córdoba}{Facultad de
  Matemática, Astronomía, Física y Computación}{}{\textit{GPA -- 9.9
    out of 10.0}}

\cventry{2019}
{Diplomarbeit}
{Formale Überprüfung der dynamischen Modallogik}
{Berater: Beta Ziliani \& Raul Fervari}
{\url{https://frantrucco.github.io/2019/04/07/thesis}}
{}

\section{Berufserfahrung}
\cventry{Marz 2021 -- Vorhanden}{Doktorand}{Max Planck Institute
for Security and Privacy}{}{}{}

\cventry{Marz 2020 -- Feb 2021}{Projektassistent}{TU Wien,
Institut of Logic and Computation,
Security and Privacy}{}{}{}

\cventry{Sept 2019 -- Jan 2020}{Maschinelles Lernen Ingenieur}{Deepvisionai}{}{}{\url{www.deepvisionai.com}}

\cventry{Jan 2017 -- Feb 2017}{Lehrassistent für Aufnahmekurs}{Facultad de Matemática, Astronomía, Física y Computación,
  Universidad Nacional de Córdoba}{Córdoba, Argentinien}{}{}

\cventry{Marz 2017 -- Feb 2020}{Lehrerassistent}{Facultad de Matemática, Astronomía, Física y Computación,
  Universidad Nacional de Córdoba}{Córdoba, Argentinien}{}{}

\section{Kurse und Schulen}

\cvitem{2020}{Coq Andes Summer School 2020.}

\cvitem{2019}{Cornell, Maryland, Max Planck Pre-doctoral Research
  School (\textbf{CMMRS}) 2019. Max Planck Institute for Software
  Systems, Cornell University and the University of Maryland.}

\cvitem{2019}{Sprecher auf ``Encuentro del grupo de investigadores
  de Lenguajes de Programación'' at the ``Universidad Nacional de
  Quilmes''. Internationaler Workshop. Buenos Aires, Argentinien}

\cvitem{2018}{``Encuentro del grupo de investigadores de Lenguajes de
  Programación'' at the ``Centro Internacional Franco Argentino de Ciencias de
  la Informacion y de Sistemas (CIFASIS)''. Internationaler Workshop. Rosario,
  Argentinien}

\cvitem{2016}{Primera Escuela de Primavera del Capítulo Argentino de la
    IEEE-GRSS. Instituto de Altos Estudios Espaciales Mario Gulich (CONAE/UNC),
    IEEE-Geoscience and Remote Sensing Society, CONAE and FaMAF.}

\cvitem{2016}{Escuela de Ciencias Informáticas. Departamento de Computación de
  la Facultad de Ciencias Exactas y Naturales, Universidad de Buenos Aires.}

\cvitem{2014, 2015}{Training Camp Argentina for ICPC Regionals. Departamento de
  Computación de la Universidad de Buenos Aires -- Facultad de Ciencias Exactas
  y Naturales}

\section{Publikationen}

\cvitem{2021}{R. Fervari, F. Trucco, B. Ziliani. Verification of dynamic bisimulation theorems in Coq.
Journal of Logical and Algebraic Methods in Programming Volume 120, April 2021.}

\cvitem{2019}{R. Fervari, F. Trucco, B. Ziliani. Mechanizing Bisimulation
  Theorems for Relation-Changing Logics in Coq. Dali 2019 (Konferenz)}

\cvitem{2018}{J. M. Scavuzzo, F. Trucco, M. Espinosa, C. B. Tauro,
  M. Abril, C. M. Scavuzzo, A. C. Frery. Modeling Dengue Vector
  Population Using Remotely Sensed Data and Machine
  Learning. Acta Tropica.}

\cvitem{2017}{Juan M. Scavuzzo, Francisco C. Trucco, Carolina Tauro,
  Alba German, Manuel Espinosa, Marcelo Abril. Modelando el patrón
  temporal del vector de dengue, Chikungunya y Zika a partir de
  información satelital con redes neuronales. RPIC IEEE. (Konferenz)}

\section{Wettbewerbe}

\cvitem{2015, 2016}
{ACM ICPC Latin America Regional}

\cvitem{2014, 2015, 2016, 2017}
{Torneo Argentino de Programación (Erste Runde der ACM-ICPC)}

\section{Fähigkeiten}

\cvitem{Programmier- sprachen}{
  \textsc{Coq},
  \textsc{LaTeX},
  \textsc{OCaml},
  \textsc{Haskell},
  \textsc{Rust},
  \textsc{C},
  \textsc{Python},
  \textsc{R},
  \textsc{MATLAB},
  \textsc{Javascript}
}

\cvitem{Andere}{
  \textsc{Git},
  \textsc{SVN},
  \textsc{Docker},
  GNU/Linux environment
}

\cvitem{Sprachen}{
  Spanish (Native),
  English (Advanced),
  German (Basic)
}

\end{spacing}
\end{document}
