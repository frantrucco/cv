\documentclass[10pt,a4paper,roman]{moderncv}

\moderncvstyle{classic}
% Options: casual, classic, oldstyle and banking
\moderncvcolor{purple}
% Options: blue, orange, green, red, purple, grey and black

\usepackage[pdftex,top=1.0cm,left=2.5cm,right=2.5cm,bottom=1.5cm]{geometry}
\usepackage[utf8]{inputenc}
\usepackage{setspace}

\firstname{Francisco}
\familyname{Trucco}

\title{Curriculum Vitae}
% \address{Santa Rosa 2004, 2 ``B''}{Córdoba Capital, Córdoba, Argentina}
% \mobile{+43 0664 4666 289}

\email{franciscoctrucco@gmail.com}

\begin{document}

\makecvtitle

\begin{spacing}{1.05}
\section{Education}

\cventry{2014 -- December 2019}{Licenciatura en Ciencias de la
  Computación (equivalent to Bachelor of Computer Science + Master of
  Computer Science.)}{Universidad Nacional de Córdoba}{Facultad de
  Matemática, Astronomía, Física y Computación}{}{\textit{GPA -- 9.9
    out of 10.0}}

\cventry{2019}
{Master Thesis}
{Formal Verification of Dynamic Modal Logics}
{Advisors: Beta Ziliani \& Raul Fervari}
{An available pdf (in spanish) can be found in the following link}
{\url{https://frantrucco.github.io/2019/04/07/thesis}}

\cventry{2014 -- 2017}{Analista en Computación (equivalent to Bachelor
  of Computer Science.)}{Universidad Nacional de Córdoba}{Facultad de
  Matemática, Astronomía, Física y Computación}{}{\textit{GPA -- 9.8
    out of 10.0}}


\section{Work Experience}
\cventry{Mar 2019 -- Ongoing}{Project Assistant}{TU Wien,
Institut of Logic and Computation,
Security and Privacy}{}{}{}

\cventry{Sep 2019 -- Jan 2020}{Machine Learning
  Engineer}{Deepvisionai}{}{}{\url{www.deepvisionai.com}}

\cventry{Jan 2017 -- Feb 2017}{Entrance course Teacher
  Assistant}{Facultad de Matemática, Astronomía, Física y Computación,
  Universidad Nacional de Córdoba}{Córdoba, Argentina}{}{}

\cventry{Mar 2017 -- Feb 2020}{Computer Science Teacher
  Assistant}{Facultad de Matemática, Astronomía, Física y Computación,
  Universidad Nacional de Córdoba}{Córdoba, Argentina}{}{}

\section{Courses and Schools}

\cvitem{2020}{Coq Andes Summer School 2020.}

\cvitem{2019}{Cornell, Maryland, Max Planck Pre-doctoral Research
  School (\textbf{CMMRS}) 2019. Max Planck Institute for Software
  Systems, Cornell University and the University of Maryland.}

\cvitem{2019}{Speaker at the ``Encuentro del grupo de investigadores
  de Lenguajes de Programación'' at the ``Universidad Nacional de
  Quilmes''. International workshop (Encuentro). Buenos
  Aires, Argentina}

\cvitem{2018}{Attendee at the ``Encuentro del grupo de investigadores
  de Lenguajes de Programación'' at the ``Centro Internacional Franco
  Argentino de Ciencias de la Informacion y de Sistemas
  (CIFASIS)''. International Workshop (Encuentro). Rosario, Argentina}

\cvitem{2016}{Primera Escuela de Primavera del Capítulo Argentino de la
    IEEE-GRSS. Instituto de Altos Estudios Espaciales Mario Gulich
    (CONAE/UNC), IEEE-Geoscience and Remote Sensing Society, CONAE
    and FaMAF.}

\cvitem{2016}{Escuela de Ciencias Informáticas. Departamento de
  Computación de la Facultad de Ciencias Exactas y Naturales,
  Universidad de Buenos Aires.}

\cvitem{2014, 2015}{Training Camp Argentina for ICPC
  Regionals. Departamento de Computación de la Universidad de Buenos
  Aires -- Facultad de Ciencias Exactas y Naturales}

\section{Publications}

\cvitem{2019}{R. Fervari, F. Trucco, B. Ziliani. Mechanizing
  Bisimulation Theorems for Relation-Changing Logics in Coq. Presented
  at Dali 2019 (Conference)}

\cvitem{2018}{J. M. Scavuzzo, F. Trucco, M. Espinosa, C. B. Tauro,
  M. Abril, C. M. Scavuzzo, A. C. Frery. Modeling Dengue Vector
  Population Using Remotely Sensed Data and Machine
  Learning. Published in Acta Tropica.}

\cvitem{2017}{Juan M. Scavuzzo, Francisco C. Trucco, Carolina Tauro,
  Alba German, Manuel Espinosa, Marcelo Abril. Modelando el patrón
  temporal del vector de dengue, Chikungunya y Zika a partir de
  información satelital con redes neuronales. RPIC IEEE. (Conference)}

\section{Competitions}

\cvitem{2015, 2016}
{ACM ICPC Latin America Regional}

\cvitem{2014, 2015, 2016, 2017}
{Torneo Argentino de Programación (First round of ACM-ICPC in Argentina)}

\section{Skills}

\cvitem{Programming Lenguages}{
  \textsc{Coq},
  \textsc{LaTeX},
  \textsc{OCaml},
  \textsc{Haskell},
  \textsc{Rust},
  \textsc{C},
  \textsc{Python},
  \textsc{R},
  \textsc{MATLAB},
  \textsc{Javascript}
}

% \cvitem{Libraries}{
%   \textsc{Scikit-Learn},
%   \textsc{NumPy},
%   \textsc{Pandas},
%   \textsc{Matplotlib},
%   \textsc{OpenCV},
%   \textsc{Seaborn},
% }

\cvitem{Others}{
  \textsc{Git},
  \textsc{SVN},
  \textsc{Docker},
  GNU/Linux environment
}

\cvitem{Natural Languages}{
  Spanish (Native),
  English (Advanced),
  German (Basic)
}

\end{spacing}
\end{document}
