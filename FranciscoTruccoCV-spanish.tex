\documentclass[10pt,a4paper,roman]{moderncv}

\moderncvstyle{classic}
% Options: casual, classic, oldstyle and banking
\moderncvcolor{red}
% Options: blue, orange, green, red, purple, grey and black

\usepackage[pdftex,top=1.0cm,left=2.5cm,right=2.5cm,bottom=1.5cm]{geometry}
\usepackage[utf8]{inputenc}
\usepackage{setspace}

\firstname{Francisco}
\familyname{Trucco}

\title{Curriculum Vitae}
\address{Santa Rosa 2004, 2 ``B''}{Córdoba Capital, Córdoba, Argentina}
\mobile{+54 351 3607668}
\email{franciscoctrucco@gmail.com}


\begin{document}

\makecvtitle

\begin{spacing}{1.05}
\section{Educación}

\cventry{2014 -- Septiembre 2019}{Licenciatura en Ciencias de la
  Computación}{Universidad Nacional de Córdoba}{FaMAF}{}
{
  % \begin{itemize}
  %       \item Análisis Matematico I (10)
  %       \item Introducción a los Algoritmos (10)
  %       \item Matemática Discreta I (9)
  %         %
  %       \item Álgebra (10)
  %       \item Análisis Matematico II (10)
  %       \item Algoritmos y Estructuras de Datos I (10)
  %         %
  %       \item Análisis Numérico (10)
  %       \item Algoritmos y Estructuras de Datos II (10)
  %       \item Organización del Computador (9)
  %         %
  %       \item Introducción a la Lógica (10)
  %       \item Probabilidad y Estadística (9)
  %       \item Sistemas Operativos (10)
  %         %
  %       \item Matemática Discreta II (10)
  %       \item Paradigmas de la Programación (10)
  %       \item Redes y sistemas Distribuídos (10)
  %         %
  %       \item Arquitectura del Computador (10)
  %       \item Bases de Datos (10)
  %       \item Ingeniería del Software I (10)
  %         %
  %       \item Lenguages Formales y Computabilidad (10)
  %       \item Modelos y Simulación (10)
  %         %
  %       \item Física (10)
  %       \item Lógica (10)
  %         %
  %       \item Lenguages Formales y Compiladores (10)
  %       \item Análisis Estadístico de Imágenes Satelitales (10)
  %       \item Procesamiento de Lenguajes naturales (10)
  %         %
  %       \item Trabajo Especial (10)
  %   \end{itemize}
    \textit{Promedio -- 9.88 de 10.0} \\
}

\cventry{2014 -- 2017}{Analista en Computación}{Universidad Nacional de
  Córdoba}{FaMAF}{}{\textit{Promedio -- 9.83 de 10.0}}

\cventry{2019}
{Trabajo Especial para la Licenciatura en Ciencias de la Computación}
{Verificación Formal de Lógicas Dinámicas en Coq}
{Directores: Beta Ziliani y Raul Fervari}
{}
{\url{https://frantrucco.github.io/2019/04/07/thesis/}}

\section{Investigación}

\cvitem{2018-Presente}{Miembro del equipo de investigación ``Logics,
  Interaction and Intelligent Systems'' (LIIS), FAMAF, Universidad
  Nacional de Córdoba, Argentina.}

\section{Experiencia Laboral}

\cventry{Marzo 2019 -- Presente}{Freelancer}{Upwork}{}{}{}

\cventry{Marzo 2017 -- Presente}{Ayudante Alumno en Ciencias de la
  Computación}{Facultad de Matemática, Astronomía, Física y
  Computación, Universidad Nacional de Córdoba}{Córdoba,
  Argentina}{}
{Materias:
  Introducción a los Algoritmos,
  Sistemas Operativos,
  Matemática Discreta II,
  Introducción a la Lógica y
  Paradigmas de los Lenguajes de Programación
}

\cventry{Enero 2017 -- Febrero 2017}{Ayudante Alumno en el Cursillo de
Nivelación}{Facultad de Matemática, Astronomía, Física y Computación,
  Universidad Nacional de Córdoba}{Córdoba, Argentina}{}{}

\section{Becas}

\cvitem{Agosto 2019}{Beca para asistir a la Escuela de Verano
  Predoctoral ``Cornell, Maryland, Max Planck Pre-doctoral Research
  School 2019''}

\cvitem{Abril 2017 -- Diciembre 2017}{Ayudante Alumno de
  Extensión. Instituto Técnico Superior Córdoba en conjunto con la
  Facultad de Matemática, Astronomía, Física y Computación,
  Universidad Nacional de Córdoba. Córdoba, Argentina}

\cvitem{Agosto 2018 -- Presente}{Tutor en el Proyecto de Mejoramiento
  de la Enseñanza en Carreras de Informática (PROMINF). Facultad de
  Matemática, Astronomía, Física y Computación, Universidad Nacional
  de Córdoba. Córdoba, Argentina}

\cvitem{Marzo 2014 -- Marzo 2019}{Becas Académicas terciarias y
  universitarias. Beca para la formación académica en la Facultad de
  Matemática, Astronomía, Física y Computación, Universidad Nacional
  de Córdoba. Financiada por la Secretaría de Equidad y Promoción del
  Empleo, Gobierno de la Provincia de Córdoba. Córdoba, Argentina}

\section{Proyectos de Investigación}
\cvitem{Marzo 2019 -- Septiembre 2020}{Formalización y Verificación
  Interactiva de Lógicas Modales Dinámicas. Grupo de Reciente
  Formación con Tutores. Financiación de 35.000 pesos otorgada por el
  Ministerio de Ciencia y Tecnologia del Gobierno de la Provincia de
  Córdoba. Función desempeñada: Estudiante.}

\section{Participación en Eventos}
\cvitem{2019}{Encuentro del grupo de investigadores de Lenguajes de
  Programación en la Universidad Nacional de Quilmes. Encuentro
  Internacional. Asistente y Conferencista. Buenos Aires, Argentina}

\cvitem{2018}{Encuentro del grupo de investigadores de Lenguajes de
  Programación en el Centro Internacional Franco Argentino de Ciencias
  de la Informacion y de Sistemas (CIFASIS). Encuentro
  Internacional. Asistente. Rosario, Argentina}

\cvitem{2017}{XVIII Workshop on Information Processing and
  Control. Asistente y Conferencista. Mar del Plata, Argentina.}

\cvitem{2016}{Primera Escuela de Primavera del Capítulo Argentino de la
    IEEE-GRSS. Instituto de Altos Estudios Espaciales Mario Gulich
    (CONAE/UNC), IEEE-Geoscience and Remote Sensing Society, CONAE
    and FaMAF. Asistente.}

\cvitem{2016}{Escuela de Ciencias Informáticas. Departamento de
  Computación de la Facultad de Ciencias Exactas y Naturales,
  Universidad de Buenos Aires. Asistente.}

\cvitem{2014, 2015}{Training Camp Argentina for ICPC
  Regionals. Departamento de Computación de la Universidad de Buenos
  Aires -- Facultad de Ciencias Exactas y Naturales. Asistente.}

\section{Publicaciones}

\cvitem{2019}{Raul Fervari, Francisco Trucco, Beta
  Ziliani. Mechanizing Bisimulation Theorems for Relation-Changing
  Logics in Coq. DALI 2019. (Conferencia)}

\cvitem{2018}{J. M. Scavuzzo, F. Trucco, M. Espinosa, C. B. Tauro,
  M. Abril, C. M. Scavuzzo, A. C. Frery. Modeling Dengue Vector
  Population Using Remotely Sensed Data and Machine
  Learning. Publicado en Acta Tropica.}

\cvitem{2017}{Juan M. Scavuzzo, Francisco C. Trucco, Carolina Tauro,
  Alba German, Manuel Espinosa, Marcelo Abril. Modelando el patrón
  temporal del vector de dengue, Chikungunya y Zika a partir de
  información satelital con redes neuronales. RPIC IEEE. (Conferencia)}

\section{Gestión Universitaria}

\cvitem{Junio 2018 -- Junio 2019}{Consejero por el claustro
  estudiantil en el Consejo Directivo de la Facultad de Matemática,
  Astronomía, Física y Computación, Universidad Nacional de Córdoba}

\cvitem{Junio 2017 -- Junio 2018}{Consejero por el claustro
  estudiantil en el Consejo Directivo de la Facultad de Matemática,
  Astronomía, Física y Computación, Universidad Nacional de Córdoba}

\section{Competencias}

\cvitem{2015, 2016}
{ACM ICPC Latin America Regional}

\cvitem{2014, 2015, 2016, 2017}
{Torneo Argentino de Programación (Primera ronda de las ACM-ICPC en Argentina)}

\section{Cursos}

\cvitem{2016}{Análisis Estadístico de Imágenes Satelitales. Aprobado
  con 10.}

\cvitem{2019}{Procesamiento de Lenguaje Natural. Aprobado con 10.}

\section{Conocimientos de Programación}

\cvitem{Lenguajes}{
  \textsc{Python},
  \textsc{R},
  \textsc{MATLAB},
  \textsc{Bash},
  \textsc{C},
  \textsc{OCaml},
  \textsc{Haskell},
  \textsc{Coq},
  \textsc{LaTeX},
  \textsc{Javascript},
}

\cvitem{Librerías y Frameworks}{
  \textsc{Scikit-Learn},
  \textsc{NumPy},
  \textsc{Pandas},
  \textsc{Matplotlib},
  \textsc{OpenCV},
  \textsc{Seaborn}
}

\cvitem{Otros}{\textsc{Git}, GNU/Linux}

\section{Languages}

\cvitemwithcomment{Español}{Nativo}{}
\cvitemwithcomment{Inglés}{Fluído}{}

\end{spacing}
\end{document}
